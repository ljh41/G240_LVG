\section{Summary}\label{summary}

We have presented the first CO multi-transition (CO 2-1, 3-2, 6-5, 7-6) study towards the molecular outflow of the high-mass star-forming region \objectname{G240}. The morphologies seen in the emission maps of four lines are very similar.  The line ratios of the four transitions are remarkably constant with the outflow velocity.  With the LVG analysis, we have constrained the temperatures to $\sim$ 50 K and found a decreasing trend of CO column density with gas velocity. We also constrain the H$_2$ density to values higher than $n \sim 10^5$ cm$^{-3}$ and found that the outflowing gas is thermalized. With the RD analysis, we found similar results of the LVG analysis in the temperature and the CO column density. The T-V relation of the G240 outflow agrees well with the wide-angle wind-driven model. Assuming a constant CO abundance ratio and a constant velocity gradient, we detect a decreasing gas density with velocity. A decreasing gas density with velocity is also consistent with the wide-angle wind-driven model. We conclude that the wide-angle wind-driven entrainment is the dominant driving mechanism of the G240 outflow. This finding indicate that disk-accretion can be responsible for the formation of high-mass stars even more massive than early B-type stars.

