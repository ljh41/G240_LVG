\section{INTRODUCTION}
Bipolar molecular outflows, mostly observed via CO, HCO$^+$ and their isotopes, are a common phenomenon associated with young stellar objects (YSOs) of all masses\citep{ 1985ARA&A..23..267L,1993prpl.conf..603F, 2001ApJ...552L.167Z,2002A&A...383..892B, 2004A&A...426..503W, 2005AJ....129..330W,  2015MNRAS.453..645M}. In sites of low-mass star formation, molecular outflows involve amounts of energy similar to those involved in the accretion processes and are indispensable for dissipating excess angular momentum of accretion disks \citep{1987ARA&A..25...23S, 1996ARA&A..34..111B}. Given the fact that outflows can continuously inject momentum into supersonic turbulence to keep the cloud from collapsing, outflows are essential to explain the low star formation efficiency in turbulent clouds. outflows also play a key role in determining the core-to-star efficiencies \citep{2014prpl.conf..451F}. Thus, molecular outflows are a fundamental part of the formation process of low mass stars. 

The driving mechanism of molecular outflows remains unknown. Molecular outflows from low-mass protostars were thought to be entrained by a wide-angle wind \citep{1991ApJ...370L..31S, 2001ApJ...557..429L}, or by a jet bow shock \citep{ 1993A&A...278..267R, 1993ApJ...414..230M, 2001ApJ...557..429L}. In the wind-driven model, a molecular outflow is the ambient material swept-up by a wide-angle radial wind. In the jet-driven model, a molecular outflow is accelarated by a jet bow shock when a jet propagates into the ambient gas. Though these models predict many different outflow properties \citep{2007prpl.conf..245A}, most of the previous observational work has focused on comparing the morphology and the kinematics of outflows with the model predictions. The shapes of molecular outflows range from parabolic molecular shells in RNO 91 and VLA 05487 \citep{2000ApJ...542..925L}, which is consistent with the wind-driven model, to the bow-shaped shells in HH211 \citep{1999A&A...343..571G} and HH 212 \citep{2000ApJ...542..925L, 2015ApJ...805..186L}, which is consistent with the jet-driven model. There are also outflows where the two diffferent features exist simultaneously: HH 111 \citep{2000ApJ...542..925L}, HH 46/47 \citep{2013ApJ...774...39A, 2016ApJ...832..158Z}, IRAS 04166+2706 \citep{2009A&A...495..169S}.
Between the wind-driven model and the jet-driven model, their predictions on the P-V diagrams cut along the outflow axis can be clearly distinguished: a convex spur structure with the highest velocity at the bow tip as predicted by the jet-driven model and a parabolic structure originating from the central source as predicted by the wind-driven model \citep{2001ApJ...557..429L}. The P-V relations of some CO outflows are best explained by the jet-driven model, such as the outflows of HH 212 and HH 240/241 \citep{2000ApJ...542..925L}. And the P-V relations of some other CO outflows are best explained by the wind-driven model, such as the outflows drived by IRAS 16293-2422 \citep{2008ApJ...675..454Y} and VLA 05847+0255 \citep{2000ApJ...542..925L}. On the other hand, the P-V structure of some CO outflows need to be explained by a combination of a jet-driven model and a wind-driven model, such as the HH 111 outflow \citep{2000ApJ...542..925L} and the HH 46/47 outflow \citep{2013ApJ...774...39A}. The wind-driven model and the jet-driven model also show a difference between the power-law index of the M-V relation. However, different simulations of each model have yielded different slopes for the computed CO intensity-velocity relations, making it difficult to compare the observed power-law index with the model predictions. For a outflow drived by a low-mass YSO, the wind-driven models generally predict a smaller intensity-velocity power-law index over a narrower range when compared to the jet-driven models \citep{2007prpl.conf..245A}. Though the wind-driven model and the jet-driven model can each explain the characteristics of some observed outflows, none of them are capable of producing the observed features of all types of outflows \citep{2002ApJ...576..294L}. To explain the two types of outflows simultaneously, two-component models with both highly collimated jet and wide-angle wind have been developed \citep{2000prpl.conf..789S, 2006ApJ...641..949B, 2006MNRAS.365.1131P, 2006ApJ...649..845S, 2007prpl.conf..277P, 2008ApJ...676.1088M, 2017arXiv171100384M}.

Due to the rarity and typically larger distances, the case of massive molecular outflows is more problemtic than their low-mass counterparts. And there is little work on modeling outflows from high-mass YSOs. Observations have shown that the masses, momenta, sizes and energies of molecular outflows from massive YSOs are higher than their low-mass counterparts \citep{2002A&A...383..892B, 2005AJ....129..330W}, and the morphology and kinematics of some outflows driven by massive YSOs are very similar to the outflows driven by low mass YSOs \citep[][]{1998ApJ...507..861S, 2002A&A...387..931B, 2008A&A...485..137C, 2009ApJ...696...66Q, 2011MNRAS.415L..49R, 2012ApJ...744L..26S}. However, with the currently small sample of individual studies towards outflows driven by massive YSOs, many questions, e.g., how the outflows from massive YSOs are accelerated, how they differ from low-mass outflows, and how they affect the high-mass star-forming processes, are still unanswered. It is essential to address these questions by studying more individual outflows associated with high-mass star-forming regions. 

Most previous studies of outflows have used low-J rotational transitions of CO (transitions up to J$_u$ = 3, with upper-state energies E$_u$ $<$ 30 K), which are easily excited at low temperatures, to characterize the relatively cold and extended molecular gas in morphology and kinematics  These low-J CO emission lines can be easily observed from the ground-based facilities. Due to atmospheric limits, observations of mid-J CO lines (referring to CO (6-5) and CO (7-6) throughout this paper), which are less affected by ambient gas and can probe the warm and/or dense gas, are very rare. To derive the physical parameters of the outflowing gas, sensitive observations of CO across multiple transitions are needed. 

%\citep{1998ApJ...507..861S, 2004ApJ...604..258S, 2008ApJ...675..454Y, 2009ApJ...696...66Q, 2009A&A...495..169S, 2013ApJ...774...39A, 2016ApJ...832..158Z}.

This paper is a follow-up study of the \objectname{G240.31+0.07} (hereafter \objectname{G240}) outflow \citep{2009ApJ...696...66Q}. Here we report the 12-m submillimeter Atacama Pathfinder Experiment Telescope\footnote{    The Atacama Pathfinder Experiment Telescope is a collaboration between the Max-Planck-Institut f$\ddot{\mathrm{u}}$r Radioastronomie, the European Southern Observatory, and the Onsala Space Observatory.} (APEX) observations of \objectname{G240}, an active high-mass star-forming region which is associated with the young stellar object (YSO) \objectname{IRAS 07427-2400} and located at a distance of 5.41 kpc \citep{2015PASJ...67...69S}. It harbors an ultracompact HII region and is associated with OH and H$_2$O masers \citep{1993AJ....105.1495H,1997MNRAS.289..203C,1998AJ....116.1897M,1999ApJS..123..487M,2003MNRAS.341..551C}. Its far-infrared luminosity of 10$^{4.7}$ L$_\sun$ is consistent with a spectral type O8.5 zero-age main-sequence star \citep{1998AJ....116.1897M}. A near-infrared study has found two bright elongated H$_2$ emission knots near the source \citep{2002ApJ...576..313K}. \citet{2003A&A...412..175K} further argued that the shocked H$_2$ emission indicates the presence of a massive rotating disk/envelope around the luminous YSO \objectname{IRAS 07427-2400}. 

There are also millimeter and centimeter radio continuum observations toward \objectname{G240}. Two clumps were detected by \citet{2007ApJ...654L..87C} at 654 GHz (460 $\mu$m), with clump A coinciding with a VLA 6 cm point source \citep{1993AJ....105.1495H} and an H$_2$O maser. \citet{2009ApJ...696...66Q} presented a high resolution interferometric study at 1.3 mm and resolved the central part of \objectname{G240} into three dusty cores MM1, MM2, and MM3, with the brightest core MM1 coinciding with the VLA 6 cm point source spatially. \citet{2011AJ....142..147T} presented observations at 1.3, 3.6, and 6 cm and reported radio continuum emissions at the position of the three millimeter sources in at least one wavelength.

\objectname{G240} has also been mapped with single dish and interferometric observations in CO emission. High-velocity CO (1-0) gas was detected towards \objectname{G240} \citep{1991AJ....101.1435M,1996ApJ...457..267S}, tracing a bipolar outflow \citep{1996ApJ...457..267S}. \citet{1997PhDT........21H} mapped the CO (3-2) emission with a 20${\arcsec}$ beam and found a prominent bipolar outflow at a position angle (PA) of 138${\degr}$ and a weaker component at PA $\sim$ 101${\degr}$. \citet{2003A&A...412..175K} also detected a prominent component and a weaker component of the bipolar CO (3-2) outflow with a 20${\arcsec}$ beam and reported the PA of the prominent component to be 132${\degr}$. Recently, \citet{2009ApJ...696...66Q} presented a detailed single dish and interferometric study of $^{12}$CO (2-1) and $^{13}$CO (2-1) emission and detected a bipolar, wide-angle, quasi-parabolic molecular outflow. 

In addition, \citet{2014ApJ...794L..18Q} reported the detection of an hourglass magnetic field aligned within 20${\degr}$ of the outflow axis.

In this paper, we present a CO multi-transition (2-1, 3-2, 6-5, 7-6) study towards the \objectname{G240} outflow. With large velocity gradient (LVG) calculations and rotation diagram (RD) analysis, we estimate the physical parameters of the outflow as a function of gas velocity. We then discuss the results of the analysis.


