\section{INTRODUCTION}
Bipolar molecular outflows are ubiquitous around low-mass and high-mass young stellar objects \citep{ 1985ARA&A..23..267L,1993prpl.conf..603F,2001ApJ...552L.167Z,2002A&A...383..892B,2015MNRAS.453..645M}. Theoretical work suggests that, in low-mass star formation, molecular outflows are related to disk-accretion process, and they play an important role in dissipating excess angular momentum of the infalling material \citep{1987ARA&A..25...23S, 1996ARA&A..34..111B}. However, due to the rarity and the typically larger distances, massive molecular outflows and their driving sources are not well studied as their low-mass counterparts. Thus, it is not clear how the massive outflows are accelerated, how they differ from low-mass outflows, and how they affect the star forming process. It is essential to address these questions by studying individual high-mass star forming regions. 

Most previous studies of outflows have used low-excitation lines of CO, which are easily excited at low temperatures, to characterize relative cold molecular gas in morphology and kinematics \citep{2009ApJ...696...66Q,  2009ApJ...702L..66Q, 2011ApJ...728....6Q}. Due to the lack of high-J CO line observations, multi CO line studies of outflow are very rare. 

Here we report the 12-m submillimeter Atacama Pathfinder Experiment Telescope\footnote{    The Atacama Pathfinder Experiment Telescope is a collaboration between the Max-Planck-Institut f"ur Radioastronomie, the European Southern Observatory, and the Onsala Space Observatory.} (APEX) observations towards \objectname{G240.31+0.07} (hereafter \objectname{G240}), an active high-mass star formation region which is associated with the IRAS 07427-2400 source and located at a distance of 5.41 kpc \citep{2015PASJ...67...69S}. It harbors an ultracompact (UC) HII region and is reported to have OH and H$_2$O maser emission \citep{1993AJ....105.1495H,1997MNRAS.289..203C,1998AJ....116.1897M,1999ApJS..123..487M,2003MNRAS.341..551C}. Its far-infrared luminosity of 10$^{4.7}$ L$_\sun$ is consistent with a O8.5 spectral type zero-age main-sequence (ZAMS) star \citep{1998AJ....116.1897M}. A near-infrared study have found two bright elongated H$_2$ emission knots near the source \citep{2002ApJ...576..313K}. \citet{2003A&A...412..175K} further argued that the shocked H$_2$ emission indicates the presence of a massive rotating disk or envelope around the luminous YSO IRAS 07427-2400. There are also millimeter and centimeter radio continuum observations towards \objectname{G240}. Two clumps were detected by \citet{2007ApJ...654L..87C} at 654 GHz (460 $\mu$m), with clump A coinciding with a VLA 6 cm point source \citep{1993AJ....105.1495H} and an H$_2$O maser. \citet{2009ApJ...696...66Q} presented a high resolution interferometric study at 1.3 mm and resolved the central part of \objectname{G240} into three dusty cores MM1, MM2, and MM3, with the brightest core MM1 coinciding with the VLA 6 cm point source spatially. \citet{2011AJ....142..147T} presented observations at 1.3, 3.6, and 6 cm and reported radio continuum emission at the position of the three millimeter sources in at least one wavelength. \objectname{G240} have also been mapped with single dish and interferometric observations in CO emission. High velocity CO (1-0) gas was detected towards \objectname{G240} \citep{1991AJ....101.1435M,1996ApJ...457..267S}, tracing a bipolar outflow \citep{1996ApJ...457..267S}. \citet{1997PhDT........21H} mapped the CO (3-2) emission with a 20${\arcsec}$ beam and found a prominent bipolar outflow at a position angle (PA) of 138${\degr}$ and a weaker component at PA $\sim$ 101${\degr}$. \citet{2003A&A...412..175K} also detected a prominent component and a weaker component of the bipolar CO (3-2) outflow with a 20${\arcsec}$ beam and reported the PA of the prominent component to be 132${\degr}$. Recently, \citet{2009ApJ...696...66Q} presented a detailed single dish and interferometric study of $^{12}$CO(2-1) and $^{13}$CO(2-1) emission and detected a bipolar, wide-angle, quasi-parabolic molecular outflow. In addition, \citet{2014ApJ...794L..18Q} reported the detection of an hourglass magnetic field aligned within 20${\degr}$ of the outflow axis.

In this paper we present a CO multi transition (3-2, 6-5, 7-6) study towards the \objectname{G240} outflow. With large velocity gradient (LVG) calculations and population diagram analysis, we explore the physical properties of the outflowing gas as a function of outflow velocity. We then discuss the results of the analysis.


