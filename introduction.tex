\section{INTRODUCTION}
Bipolar molecular outflows, mostly observed via CO, HCO$^+$ and their isotopes, are a common phenomenon associated with young stellar objects (YSOs) of all masses \citep{2001ApJ...552L.167Z, 2002A&A...383..892B, 2004A&A...426..503W, 2005AJ....129..330W,  2015MNRAS.453..645M}. It is believed that molecular outflows trace the accretion-powered ejections in sites of low-mass star formation. As molecular outflows impact the surrounding material and the parent cloud significantly, they play an important role in the star formation process. Though molecular outflows have been well studied, the driving mechanism of molecular outflows remains unknown. Molecular outflows from low-mass protostars were thought to be entrained by wide-angle winds \citep{1991ApJ...370L..31S, 2001ApJ...557..429L}, or by jet bow shocks \citep{ 1993A&A...278..267R, 1993ApJ...414..230M, 2001ApJ...557..429L}. Though the wind-driven model and the jet-driven model can each explain the characteristics of some observed outflows, none of them are capable of producing the observed features of all types of outflows \citep{2000ApJ...542..925L, 2002ApJ...576..294L}. To explain the features of different types of outflows associated low-mass YSOs simultaneously, two-component models with both highly collimated jet and wide-angle wind have been developed \citep{2000prpl.conf..789S, 2006ApJ...641..949B, 2006MNRAS.365.1131P, 2006ApJ...649..845S, 2007prpl.conf..277P, 2008ApJ...676.1088M}.

Massive molecular outflows are more problematic than their low-mass counterparts. Though observations have shown that the morphology and kinematics of some outflows driven by massive YSOs are very similar to the outflows driven by low mass YSOs \citep[][]{1998ApJ...507..861S, 2002A&A...387..931B, 2009ApJ...696...66Q, 2011MNRAS.415L..49R}, there is a lack of detection of extremely collimated outflows and circumstellar disks towards sources more massive than early B-type YSOs \citep{2007prpl.conf..245A}. Thus, it is not clear whether massive stars form as a scaled-up version of low-mass stars or they form in a different way. Due to the rarity and typically large distances of high-mass YSOs, the sample of individual studies towards outflows driven by massive YSOs is still small. And there is little theoretical work on modeling outflows from high-mass YSOs. Many questions, e.g., how the outflows from massive YSOs are accelerated, how they differ from low-mass outflows, and how they affect the high-mass star-forming processes, are still unanswered. It is essential to address these questions by studying more outflows associated with high-mass star-forming regions. 

Most previous studies of outflows have used low-J rotational transitions of CO (transitions up to J$_u$ = 3, with upper-state energies E$_u$ up to 30 K), which are easily excited at low temperatures, to characterize the relatively cold and extended molecular gas in morphology and kinematics. These low-J CO emission lines can be easily observed from the ground-based facilities. Due to atmospheric limits, mid-J CO lines (referring to CO (6-5) and CO (7-6) throughout this paper, with  E$_u$ up to 150 K), which are less affected by ambient gas, are not commonly observed. In several studies, mid-J CO transitions have been reported to trace the warm gas (T $>$ 50 K) in outflows of low-mass and intermediate-mass YSOs \citep{2009A&A...501..633V, 2009A&A...507.1425V, 2012A&A...542A..86Y, 2016A&A...587A..17V}.  By comparing multi-line CO observations (both low-J and mid-J) with the results of non-LTE radiative transfer models, the physical properties (temperature, density) of the outflowing gas could be well constrained \citep{2015A&A...581A...4L}. 

This paper is a follow-up study of the \objectname{G240.31+0.07} (hereafter \objectname{G240}) outflow \citep{2009ApJ...696...66Q}. We report the 12-m submillimeter Atacama Pathfinder Experiment Telescope\footnote{    The Atacama Pathfinder Experiment Telescope is a collaboration between the Max-Planck-Institut f$\ddot{\mathrm{u}}$r Radioastronomie, the European Southern Observatory, and the Onsala Space Observatory.} (APEX) observations of \objectname{G240}, an active high-mass star-forming region which is associated with the YSO \objectname{IRAS 07427-2400} and located at a distance of 5.41 kpc \citep{2015PASJ...67...69S}. It harbors an ultracompact HII region and is associated with OH and H$_2$O masers \citep{1993AJ....105.1495H, 1997MNRAS.289..203C, 1998AJ....116.1897M, 1999ApJS..123..487M, 2003MNRAS.341..551C}. Its far-infrared luminosity of 10$^{4.7}$ L$_\sun$ is consistent with a spectral type O8.5 zero-age main-sequence star \citep{1998AJ....116.1897M}. \citet{2009ApJ...696...66Q} presented a high resolution interferometric study at 1.3 mm and resolved the central part of \objectname{G240} into three dusty cores MM1, MM2, and MM3. \citet{2003A&A...412..175K} mapped the CO (3-2) emission with a 20${\arcsec}$ beam and found a prominent bipolar outflow at a position angle (PA) of 132${\degr}$ and a weaker component at PA $\sim$ 101${\degr}$. Recently, \citet{2009ApJ...696...66Q} presented a detailed single dish and interferometric study of $^{12}$CO (2-1) and $^{13}$CO (2-1) emission and detected a bipolar, wide-angle, quasi-parabolic molecular outflow. In addition, \citet{2014ApJ...794L..18Q} reported the detection of an hourglass magnetic field aligned within 20${\degr}$ of the outflow axis.

In this paper, we present a CO multi-transition (2-1, 3-2, 6-5, 7-6) study towards the \objectname{G240} outflow. With large velocity gradient (LVG) calculations and rotation diagram (RD) analysis, we estimate the physical parameters of the outflow as a function of gas velocity. We then discuss the results of the analysis.


