\section{Introduction}
Bipolar molecular outflows, mostly observed via CO, HCO$^+$ and their isotopes, are a common phenomenon associated with young stellar objects (YSOs) of all masses \citep{2001ApJ...552L.167Z, 2002A&A...383..892B, 2004A&A...426..503W, 2005AJ....129..330W,  2015MNRAS.453..645M}. It is believed that molecular outflows trace the accretion-powered ejections in sites of low-mass star formation, and they play an important role in the star formation process by significant impacts on the surrounding material and the parent cloud. Although molecular outflows have been well studied, their driving mechanism remains unknown. Molecular outflows from low-mass YSOs were thought to be produced by the interaction between wide-angle winds and ambient gas \citep{1991ApJ...370L..31S, 2001ApJ...557..429L}, or by jet bow shocks \citep{1993A&A...278..267R, 1993ApJ...414..230M, 2001ApJ...557..429L}. Although the wind-driven model and the jet-driven model can both explain some observed outflow features, none of them are capable of producing all of the observed features of different types of outflows \citep{2000ApJ...542..925L, 2002ApJ...576..294L}. Two-component models with both highly collimated jet and wide-angle wind were then developed to establish a unified picture of the observed outflow features\citep{2000prpl.conf..789S, 2006ApJ...641..949B, 2006MNRAS.365.1131P, 2006ApJ...649..845S, 2007prpl.conf..277P, 2008ApJ...676.1088M}.

Massive molecular outflows, which are associated with high-mass YSOs, are less understood than their low-mass counterparts. Due to the rarity and typically large distances, there are few studies towards outflows driven by massive YSOs. Although observations have shown that the morphology and kinematics of some massive outflows are very similar to what is observed for those driven by low-mass YSOs \citep[][]{1998ApJ...507..861S, 2002A&A...387..931B, 2009ApJ...696...66Q, 2011MNRAS.415L..49R}, extremely collimated outflows and circumstellar disks have remained elusive for sources more massive than those equivalent to early B-type stars \citep{2007prpl.conf..245A}. And there is little theoretical work on modeling outflows from high-mass YSOs. Many questions, e.g., how the outflows from massive YSOs are accelerated, how they differ from low-mass outflows, and how they affect the high-mass star-forming processes, are still unanswered. It is essential to address these questions by studying more outflows associated with high-mass star-forming regions. 

Most previous studies of outflows have used low-J rotational transitions of CO (transitions up to J$_u$ = 3, with upper-state energies E$_u$ up to 30 K), which are easily excited at low temperatures and can be easily observed by ground-based facilities, to characterize the relatively cold and extended molecular gas in morphology and kinematics. Due to the Earth's atmosphere, mid-J CO lines (referring to CO 6-5 and CO 7-6 throughout this paper, with  E$_u$ up to 150 K), which are less affected by ambient gas, are not commonly observed. In several studies, mid-J CO transitions have been reported to trace the warm gas (T $>$ 50 K) in outflows of low-mass and intermediate-mass YSOs \citep{2009A&A...501..633V, 2009A&A...507.1425V, 2012A&A...542A..86Y, 2016A&A...587A..17V}. By comparing multi-line CO observations (both low-J and mid-J) with the results of radiative transfer models, the physical properties (temperature, gas density and CO column density) of the outflowing gas could be well constrained \citep{2015A&A...581A...4L}. 

This paper is a follow-up study of the \objectname{G240.31+0.07} (hereafter \objectname{G240}) outflow \citep{2009ApJ...696...66Q}. We report the 12-m submillimeter Atacama Pathfinder Experiment Telescope\footnote{    The Atacama Pathfinder Experiment Telescope is a collaboration between the Max-Planck-Institut f{\"u}r Radioastronomie, the European Southern Observatory, and the Onsala Space Observatory.} (APEX) observations of \objectname{G240}, an active high-mass star-forming region associated with \objectname{IRAS 07427-2400} and located at a distance of $\sim$5.4 kpc \citep{2014ApJ...790...99C, 2015PASJ...67...69S}. It harbors an ultracompact H{\scriptsize II} region and is associated with OH and H$_2$O masers \citep{1993AJ....105.1495H, 1997MNRAS.289..203C, 1998AJ....116.1897M, 1999ApJS..123..487M, 2003MNRAS.341..551C}. Its far-infrared luminosity of 10$^{4.7}$ L$_\sun$ is consistent with a spectral type O8.5 zero-age main-sequence star \citep{1998AJ....116.1897M}. \citet{2009ApJ...696...66Q} presented a high resolution interferometric study at 1.3 mm and resolved the central part of \objectname{G240} into three dusty cores MM1, MM2, and MM3. \citet{2003A&A...412..175K} mapped the CO 3-2 emission with a 20${\arcsec}$ beam and found a prominent bipolar outflow at a position angle (PA) of 132${\degr}$ and a weaker component at PA $\sim$ 101${\degr}$. From C$^{18}$O 2-1 observations, \citet{2003A&A...412..175K} found the cloud velocity ($v_{\mathrm{cloud}}$) with respect to the local standard of rest to be $\sim$ 67.5 km s$^{-1}$. Recently, \citet{2009ApJ...696...66Q} presented a detailed single dish and interferometric study of $^{12}$CO 2-1 and $^{13}$CO 2-1 emissions and detected a bipolar, wide-angle, quasi-parabolic molecular outflow. \citet{2013A&A...559A..23L} theoretically interpreted the G240 outflow as a result from interaction between the wind and the ambient gas in the form of turbulent entrainment. In addition, \citet{2014ApJ...794L..18Q} reported the detection of an hourglass magnetic field aligned within 20${\degr}$ of the outflow axis.

In this paper, we present a CO multi-transition (2-1, 3-2, 6-5, 7-6) study towards the \objectname{G240} outflow. With rotation diagram (RD) analysis and large velocity gradient (LVG) calculations, we estimate the physical parameters of the outflow as functions of gas velocity. We then discuss the results of the analysis.


