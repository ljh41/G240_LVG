\section{DISCUSSION}\label{discussion}
With priori assumption of other parameters, the kinetic temperature of the outflowing gas can be derived from LVG analysis using two CO lines. A rising trend of CO 3-2/6-5 ratio is observed towards the outflow gas associated with low mass YSO HH46 \citep{2009A&A...501..633V}. If the density remains constant, the rising ratios observed at more extreme velocities could correspond to lower kinetic temperatures. In the case of outflowing gas toward low mass protostars NGC 1333 IRAS 4A and  IRAS 4B, the CO 3-2/6-5 ratios are remarkably constant with velocity \citep{2012A&A...542A..86Y}. With the assumption of constant density, the constant ratio trend shows little or no evidence of a temperature change with velocity. Based on CO (2-1) and CO (3-2) observations towards the extremely high velocity outflow of high mass YSO G5.89-0.39, \citet{2012ApJ...744L..26S} assumed the canonical CO fractional abundance of 10$^{-4}$ and performed a LVG anylysis, revealing a increasing trend of temperature with gas velocity. However, using the CO (6-5), (7-6) and (16-15) lines, \citet{2015A&A...584A..70L} performed a rotation diagram analysis towards G5.89-0.39 outflow and found a decreasing trend of temperature with increasing velocities. This disagreement seen in results of \citet{2012ApJ...744L..26S} and \citet{2015A&A...584A..70L} cound be due to different angular resolutions (3${\arcsec}$.4 compared to 14${\arcsec}$.5 ). While \citet{2012ApJ...744L..26S} has smaller energy range covered by their CO line observations ($\Delta$E$_u$ $\sim$ 17 K) compared to $\Delta$E$_u$ $\sim$ 600 K of \citet{2015A&A...584A..70L}, the decreasing trend of temperature with velocity constrained in \citet{2015A&A...584A..70L} is probably more appropriate for the G5.89-0.39 outflow. The different distributions of temperature with velocity reveal the complexity of molecular outflows. 

In the wide-angle wind model, a molecular outflow is ambient material swept-up by a wide-angle radial wind. In previous numerical works, many authors have chosen isothermal equations for this model \citep{1996ApJ...472..211L,2001ApJ...557..429L}. Molecular cooling dominates the cooling of the shocked material in the outflow at temperatures below 10$^4$ K \citep{1997IAUS..182..181H}. As the cooling rate increases as n$^2$, molecular cooling is very efficient for the density of a wind-driven outflow. Thus, an isothermal state could be reached in a wind-driven outflow, with no temperature change with velocity. This is consistent with our derivation of almost constant temperature in the massive \objectname{G240} outflow. In addition, some other features of the molecular outflow associated with \objectname{G240} can also be qualitatively interpreted by the wide-angle wind model \citep{2009ApJ...696...66Q}, while other outflow models have different predictions of these features, as reviewed by \citet{2007prpl.conf..245A}. However, most outflow models existed have parameters typical of low mass outflows. It is necessary to compare the observational results of high mass outflows with models of similar physical conditions. Statics of outflows associated with high mass star forming regions are also essential for us to better understand the driven mechanism of massive outflows and the forming process of high mass stars.