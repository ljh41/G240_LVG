\section{Discussion}\label{discussion}
The morphology and kinematics of molecular outflows have been widely studied based on single spectral line observations. However, the physical conditions of the outflow gas are still not well constrained. In paticular, how the physical conditions vary with the outflow velocity is poorly understood. To precisely constrain the physical properties of molecular outflows, CO observations across a wide range of energy levels are needed. Previous studies show that CO transitions of $J_{\mathrm{up}} > 9$ (high-J) and with $J_{\mathrm{up}} \le 9$ should be fitted with different gas components, and that the hotter, denser gas component traced by high-J CO lines only constitutes a small amount of the gas in the molecular outflow \citep{2013A&A...555A...8G, 2015A&A...581A...4L}. Thus, the driven mechanism of the bulk of the molecular outflow is best studied in lines of $J_{\mathrm{up}} \le 9$. Our multi-line analysis with both low-J and mid-J CO line observations have constrained the CO column density and the temperature of the G240 outflow as functions of the outflow velocity, showing a clear example on the physical conditions of a representative well-defined bipolar wide-angle molecular outflow in a $> 10^4$ $L_\sun$ massive star-forming region \citep{2009ApJ...696...66Q}.

\subsection{Temperature}
%High sensitivity CO mid-J observations as complements of existed low-J observations are required to constrain the temperature and density of the molecular outflows as a function of velocity, while these mid-J observations are challenging because of atmospheric limits. 
%A rising CO 3--2/6--5 ratio is observed towards the outflow associated with low-mass YSO HH 46 IRS 1\citep{2009A&A...501..633V}. With the density assumed to be constant, the rising ratios at more extreme velocities correspond to lower kinetic temperatures.
%The CO 3--2/6--5 ratios of the outflow associated with low-mass protostars NGC 1333 IRAS 4A and 4B were found to be relatively constant with velocity \citep{2012A&A...542A..86Y}. With the assumption of constant densities, \citet{2012A&A...542A..86Y} suggests that the constant ratios indicate little or no temperature change with velocity. With the assumption of constant canonical CO fractional abundances and constant velocity gradients, \citet{2012ApJ...744L..26S} performed an LVG analysis on the extremely high velocity outflow associated with the high-mass YSO G5.89-0.39 in CO 2--1 and CO 3--2 and found an increasing trend of temperature with outflow velocity.

The variation of the outflow temperature with the outflow velocity have been studied only in a handful of sources. \citet{2009A&A...501..633V} and \citet{2012A&A...542A..86Y} performed LVG calculations on single-dish CO J = 3--2 and 6--5 lines, and suggested that there is little or no temperature change within outflow velocity ranges of $<$10--15 km s$^{-1}$ in the outflow associated with low-mass protostars HH46 IRS1 and NGC 1333 IRAS 4A/4B. Considering that \citet{2009A&A...501..633V} have observed rising CO 3--2/6--5 ratios at more extreme velocities, and that the line wing ratios of CO 3--2/6--5 observed by \citet{2012A&A...542A..86Y} have relatively large variations, their conclusions of ``constant'' outflow temperatures are not robust. The temperature of the outflowing gas as a function of velocity in the outflow associated with a high-mass star-forming region was first derived by \citet{2012ApJ...744L..26S}, who performed an LVG analysis on interferometer CO J = 2--1 and 3--2 observations of the extremely high velocity outflow associated with the high-mass star-forming region G5.89-0.39, and found an increasing temperature with outflow velocity for outflow velocities up to 160 km s$^{-1}$. Based on the variation of the CO 2--1/1--0 line ratio at a resolution of 32.5$\arcsec$ and assuming LTE, \citet{2018RAA....18...19X} found that the excitation temperature of the outflow in the massive star-forming region IRAS 22506+5944 increases from low ($\sim$5 km s$^{-1}$) to moderate ($\sim$8-12 km s$^{-1}$) velocities, and then decrease at higher velocities ($<$30 km s$^{-1}$). However, \citet{2012ApJ...744L..26S} and \citet{2018RAA....18...19X} have only used low-J CO lines with small energy ranges, which are not sufficient to trace the relatively warm and dense gas. Moreover, all of the above mentioned works have made assumptions about other parameters (e.g., gas density, canonical CO fractional abundance, or velocity gradient) or the equilibrium state (e.g., LTE) of the outflow to infer the $T$-$V$ relations from only two CO lines, while these assumptions might not be necessarily valid. Recently, the outflow properties were more precisely determined via multi-line CO studies by \citet{2015A&A...581A...4L}, who performed an LVG analysis on the outflow cavity ($<$40 km s$^{-1}$) of the intermediate-mass Class 0 protostar Cepheus E-mm and revealed that the outflowing gas traced by CO transitions of $J_{up} \le 9$ is nearly isothermal. Our analysis, for the first time, reveals the temperature-velocity relation in a high-mass star-forming region with both low-J and mid-J observations via the LVG calculation. The results show that the G240 outflow is approximately isothermal with a gas temperature of $\sim$50 K within an outflow velocity range of $<$23 km s$^{-1}$. This isothermal state is similar to the behavior of the outflowing gas (traced by CO transitions of $J_{up} \le 9$) associated with low-mass and intermediate-mass protostars \citep{2012A&A...542A..86Y, 2015A&A...581A...4L}, and the temperature of $\sim$50 K is slightly lower than the temperature for outflows of low-mass protostars \citep{2009A&A...501..633V, 2012A&A...542A..86Y} and intermediate-mass protostars \citep{2016A&A...587A..17V}. In addition, the derived outflow temperature is warmer than the previously adopted $\sim$30 K \citep{2009ApJ...696...66Q}, which indicates that the physical parameters (mass, momentum, energy) of the G240 outflow calculated by \citet{2009ApJ...696...66Q} were underestimated by a factor of 1.32. 

%with sophisticated radiative transfer methods (e.g., LVG analysis)
%and an increasing temperature of velocity is found in the high-excitation gas component traced by high-J CO transitions (with $J_{up}$ up to 16)
%Using the CO 6--5, 7--6 and (16-15) lines, \citet{2015A&A...584A..70L} performed a rotation diagram analysis towards the G5.89-0.39 outflow and found a decreasing trend of temperature with increasing velocities. \citet{2015A&A...584A..70L} argued that this disagreement seen in results of \citet{2012ApJ...744L..26S} and \citet{2015A&A...584A..70L} could be due to different angular resolutions and different energy range. However, with only two or three lines observed, their derived $T-V$ relations must be deduced from fixing other physical parameters, e.g., constant density or constant canonical CO fractional abundance, or from the assumption that the different transitions can be discribed with the same excitation temperature (LTE), while these assumptions might deviate from the real case.

\subsection{Density}\label{subsec:density}
From the LVG analysis, the beam-averaged CO column density is well constrained in each velocity bin, while the constraint for the gas density is loose. Figure \ref{fig:figrelation}(b) shows that, from $\pm$7 km s$^{-1}$ to $\pm$22 km s$^{-1}$ with respect to the cloud velocity, the beam-averaged CO column density decreases by a factor of 50. The decreasing $N$-$V$ trend is similar to that of the outflow cavity associated with Cepheus E-mm \citep{2015A&A...581A...4L}, and may directly indicate a decrease of the entrained gas with increasing outflow velocity. Since the CO column density degenerates with the beam filling factor in the optically thin case, the observed CO column density could be affected by the beam dilution effect. However, the beam filling factor only decreases by a factor of $\sim$2.5 (derived from source sizes of $\sim$20$\arcsec$ to $\sim$10$\arcsec$, see Figure 3 of \citet{2009ApJ...696...66Q}) within the outflow velocity range, which could not fully explain the decrease of the observed CO column density with velocity. We argue that the decrease in CO column density with outflow velocity is due to the change of the gas density. Assuming a constant [CO]/[H$_2$] abundance ratio, which is adopted by most previous works, the decreasing CO column density indicates a decline of the gas column density with velocity. If the velocity gradient do not vary much, the decreasing gas column density with velocity implies that the gas density decreases with velocity. Taken together, the variations of the beam-averaged CO column density and the beam filling factor suggest that the gas density is $>$10$^6$ cm$^{-3}$ at the lowest outflow velocity. This gas density limit is comparable to that in the outflow cavity associated with Cepheus E-mm \citep[Several times of 10$^5$ cm$^{-3}$:][]{2015A&A...581A...4L}. In summary, our observations indicate that both the CO column density and the gas denstity decrease with an increasing outflow velocity.

%the CO column density $N$ is related to the gas density $n$ through the expression: 
%\begin{equation}
%N = n \times \Delta V \times \frac{1}{dv/dr} \times X_{\mathrm{CO}} \times f_{\mathrm{b}}, 
%\end{equation}
%where $f_{\mathrm{b}}$ is the beam filling factor, $X_{\mathrm{CO}}$ the [CO]/[H$_2$] abundance ratio, $\Delta V$ the velocity interval and $dv/dr$ is the velocity gradient. 
%N = n \times \Delta V \times \frac{1}{dv/dr} \times X_{\mathrm{CO}} \times f_{\mathrm{b}}, 
%the beam-filling factor decreases from $\sim$ 0.5 down to $\sim$ 0.2 (derived from source sizes of $\sim 20\arcsec$ to $\sim 10\arcsec$, see Figure 3 of \citet{2009ApJ...696...66Q}). The beam-averaged CO column density at the lowest flow velocity is 50 times larger than that at the highest flow velocity, whereas the beam filling factor is only 2.5 times larger. The lower limit of the gas density is $\sim$ 10$^5$ cm$^{-3}$
%Such a high density could be the product of shock compression or the entrainment of a high density primary wind.
%The beam-averaged CO column density at the lowest flow velocity is 50 times larger than that at the highest flow velocity, whereas the beam filling factor is only 2.5 times larger.
%So one or serveral of these parameters ($n_{\mathrm{H}_2}$, $dv/dr$ and $X_{\mathrm{CO}}$) must have decreased with outflow velocity. 
%Due to the lack of further information, it is unclear whether the CO abundance ratio or the velocity gradient has attributed to the decrease of CO column density at higher velocities. 
%A drop in $N_{\mathrm{CO}}$ at more extreme velocities indicates the decrease of one or serveral of these parameters. 
%To explore how the effect of beam dilution influence our results, we vary the beam filling factors from 0.2 to 1 and then perform the LVG analysis again. We find that modelling with different beam filling factors mainly affect the $N$-$V$ diagram, with minor change in the $T$-$V$ diagram and density limits. This could be resulted from the degeneracies of the beam filling factor with CO column density in the optically thin case. 

\subsection{The origin of the G240 outflow}

With high-resolution CO J = 2--1 observations, \citet{2009ApJ...696...66Q} found that the G240 outflow has a well-defined, bipolar, wide-angle, quasi-parabolic morphology and a parabolic position-velocity structure in the northwest redshifted lobe. The kinematic structure and morphology are similar to low-mass wide-angle outflows. However, observations toward low-mass YSOs show that wide-angle low-velocity molecular outflows are usually accompanying with collimated high-velocity atomic/molecular jets: e.g., IRAS 04166+2706 \citep{2009A&A...495..169S}, L1448-mm \citep{2010ApJ...717...58H}, and HH 46/47 \citep{2007ApJ...668L.159V}. But the G240 region shows no signature of a high-velocity jet in infrared, millimeter, and centimeter emissions \citep{2002ApJ...576..313K, 2009ApJ...696...66Q, 2011AJ....142..147T}. Thus, it is still unclear whether the G240 outflow is driven in a way analogous to that of low-mass outflows or in a different way.

%Thus, it is still unclear whether the driven mechanism of low-mass outflows can be used to explain the G240 outflow.

To further investigate the physical conditions and to explore the driven mechanism of the G240 outflow, we presented multi-line analysis of the outflow, and revealed that the temperature of the outflowing gas is relatively constant with outflow velocity, and that the density of the outflow decreases with gas velocity. These trends are in agreement with the estimations of a simple wide-angle wind-driven model, which predict that the outflow is approximately isothermal because of efficient cooling, and that the outflow density decreases with velocity and distance from the driving source as the wind sweeps up the ambient material, which has a density inversely proportional to the square of distance from the source \citep{1991ApJ...370L..31S, 2001ApJ...557..429L}. In contrast, the jet-driven bow shock model predicts that the gas temperature and the density of the outflow increase with velocity and distance from the driving source due to shock heating and compressing \citep{2001ApJ...557..429L}, which are different from our results. The none-detection of a high-velocity jet in the G240 region also rules out the jet-driven bow shock model. Although most existing wind-driven outflow models can only explain typical parameters of outflows driven by low-mass YSOs, recent theoretical results provide evidence that massive outflows can also be driven by wide-angle disk winds \citep{2018MNRAS.475..391M}. Thus, our analyses results support the scenario that the G240 outflow is mainly driven/entrained by a wide-angle wind, which itself may resemble the accretion-driven wide-angle winds \citep[X-wind or disk winds:][]{2006ApJ...649..845S, 2006MNRAS.365.1131P} associated with low-mass YSOs. We further suggest that disk-mediated accretion may exist in the formation of high-mass stars up to late-O types. 
%and high-mass outflow models are necessary to quantitatively explain the physical conditions of outflows associated with high-mass YSOs. 
%Statistical analyses of high-mass outflows are also essential for us to better understand the driven mechanism of massive outflows and the forming process of high-mass stars.
%, although we can't differ \citep[X-wind or disk winds:][]{2006ApJ...649..845S, 2006MNRAS.365.1131P}.
%We noticed that there are faint bow-shaped H$_{2}$ features near the source \objectname{IRAS 07427-2400} \citep{2002ApJ...576..313K}, indicating that jet-driven bow shocks may also contribute to the G240 outflow. Mid-J CO observations with higher resolution (e.g., with ALMA) are needed to study these suspicious bow shock features in detail.
%As molecular cooling is very efficient for the typical density and temperature of a wind-driven outflow, an isothermal state could be reached in typical wind-driven outflows.
%As molecular cooling dominates the cooling of the shocked material in the outflow at temperatures below 10$^4$ K \citep{1997IAUS..182..181H} and the cooling rate increases as $n^2$, molecular cooling is very efficient for the typical density of a wind-driven outflow. Thus, an isothermal state could be reached in a wind-driven outflow.
