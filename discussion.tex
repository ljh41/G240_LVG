\section{Discussion}\label{discussion}
\subsection{Gas temperature}
%High sensitivity CO mid-J observations as complements of existed low-J observations are required to constrain the temperature and density of the molecular outflows as a function of velocity, while these mid-J observations are challenging because of atmospheric limits. 
%A rising CO 3-2/6-5 ratio is observed towards the outflow associated with low-mass YSO HH 46 IRS 1\citep{2009A&A...501..633V}. With the density assumed to be constant, the rising ratios at more extreme velocities correspond to lower kinetic temperatures.
%The CO 3-2/6-5 ratios of the outflow associated with low-mass protostars NGC 1333 IRAS 4A and 4B were found to be relatively constant with velocity \citep{2012A&A...542A..86Y}. With the assumption of constant densities, \citet{2012A&A...542A..86Y} suggests that the constant ratios indicate little or no temperature change with velocity. With the assumption of constant canonical CO fractional abundances and constant velocity gradients, \citet{2012ApJ...744L..26S} performed an LVG analysis on the extremely high velocity outflow associated with the high-mass YSO G5.89-0.39 in CO (2-1) and CO (3-2) and found an increasing trend of temperature with outflow velocity.

Though the morphology and kinematics of molecular outflows have been widely studied, some other physical properties (e.g., the variation of temperatures and densities with the outflow velocity and distance to the source) of molecular outflows are still unclear. In addition, studies on the the $T$-$V$ relation of molecular outflows are very few. Based on constant CO 3-2/6-5 ratios, \citet{2012A&A...542A..86Y} suggested that there is little or no temperature change with velocity in the outflow associated with low-mass protostars NGC 1333 IRAS 4A and 4B. With CO (2-1) and CO (3-2) observations, \citet{2012ApJ...744L..26S} performed an LVG analysis on the extremely high velocity outflow associated with the high-mass YSO G5.89-0.39 and found an increasing trend of temperature with outflow velocity. However, assumptions of some outflow parameters (e.g. gas density, canonical CO fractional abundance and velocity gradient) are needed to infer the $T$-$V$ relations from only two CO lines, while these assumptions might not be necessarily valid. To determine the outflow properties with better accuracy, multi-line CO studies with sophisticated radiative transfer methods (e.g. LVG analysis) are required. Recently, \citet{2015A&A...581A...4L} performed a multi-transition CO study of the outflow cavity of the intermediate-mass Class 0 protostar Cepheus E-mm and revealed an isothermal low-excitation gas component with transitions of $J_{up} \le 9$. Our analyses reveal that the G240 outflow, which is representative of a well-defined bipolar wide-angle molecular outflow in a $> 10^4 L_\sun$ star-forming region \citep{2009ApJ...696...66Q}, is isothermal and has a temperature of $\sim$ 50 K. The isothermal state is in agreement with outflows (traced by transitions of $J_{up} \le 9$) of low-mass protostars and intermediate-mass protostars \citep{2012A&A...542A..86Y, 2015A&A...581A...4L}, and the temperature value of $\sim$ 50 K is consistent with temperatures in excess of 50 K probed by \citet{2016A&A...587A..17V} for outflows associated with intermediate-mass protostars, and slightly lower than temperatures of outflows associated with low-mass protostars \citep{2009A&A...501..633V, 2012A&A...542A..86Y}. Since the derived outflow temperature of $\sim$ 50 K is warmer than previously adopted $\sim$ 30 K \citep{2009ApJ...696...66Q}, the physical parameters (mass, momentum, energy) of the G240 outflow calculated by \citet{2009ApJ...696...66Q} were underestimated by a factor of 1.32.  

%Using the CO (6-5), (7-6) and (16-15) lines, \citet{2015A&A...584A..70L} performed a rotation diagram analysis towards the G5.89-0.39 outflow and found a decreasing trend of temperature with increasing velocities. \citet{2015A&A...584A..70L} argued that this disagreement seen in results of \citet{2012ApJ...744L..26S} and \citet{2015A&A...584A..70L} could be due to different angular resolutions and different energy range. However, with only two or three lines observed, their derived $T-V$ relations must be deduced from fixing other physical parameters, e.g., constant density or constant canonical CO fractional abundance, or from the assumption that the different transitions can be discribed with the same excitation temperature (LTE), while these assumptions might deviate from the real case.

A simple wide-angle wind-driven model predicts that the gas temperature of the outflow is relatively constant with velocity and distance to the source, while different trends are predicted by other outflow models \citep{2007prpl.conf..245A}. The estimated isothermal state of the G240 outflow agrees with the prediction of the wind-driven model. As the kinematics and morphology of the G240 outflow can also be qualitatively interpreted by the wide-angle wind-driven model \citep{2009ApJ...696...66Q}, we conclude that the wide-angle wind entrainment is the dominant driving mechanism of the G240 outflow.  We noticed that there are faint bow-shaped H$_{2}$ features near the YSO \objectname{IRAS 07427-2400} \citet{2002ApJ...576..313K}, indicating that jet bow shock entrainment may also contribute to the G240 outflow. Mid-J CO observations with higher resolution (e.g. ALMA) are needed to study these suspicious bow shock features in detail. It should be noted that most existing outflow models have typical parameters of outflows driven by low-mass YSOs. High-mass outflow models are necessary to quantitatively explain the physical conditions of outflows of high-mass YSOs. Statics of massive outflows associated with high-mass star-forming regions are also essential for us to better understand the driven mechanism of massive outflows and the forming process of high-mass stars.
%As molecular cooling is very efficient for the typical density and temperature of a wind-driven outflow, an isothermal state could be reached in typical wind-driven outflows.
%As molecular cooling dominates the cooling of the shocked material in the outflow at temperatures below 10$^4$ K \citep{1997IAUS..182..181H} and the cooling rate increases as $n^2$, molecular cooling is very efficient for the typical density of a wind-driven outflow. Thus, an isothermal state could be reached in a wind-driven outflow.

\subsection{Gas density and CO column density}
%The LVG analysis reveals that the G240 outflow is thermalized and the gas densities are $> 10^5$ cm$^{-3}$. 
The beam-averaged CO column density in each 2 km s$^{-1}$ bin decreases from $\sim 2 \times  10^{16} $ cm$^{-2}$ to $\sim 4 \times 10^{14}$ cm$^{-2}$ within the outflow velocity range from $\sim \pm$ 7 km s$^{-1}$ to $\sim \pm$ 22 km s$^{-1}$ with respect to the cloud velocity. In the optically thin case, the beam-averaged CO column density $N_{\mathrm{CO}}$ is related to the gas density $n_{\mathrm{H}_2}$ through the expression: 
\begin{equation}
N_{\mathrm{CO}} = n_{\mathrm{H}_2} \times \Delta V \times \frac{1}{dv/dr} \times X_{\mathrm{CO}} \times f_{\mathrm{b}}, 
\end{equation}
where $f_{\mathrm{b}}$ is the beam filling factor, $X_{\mathrm{CO}}$ the [CO]/[H$_2$] abundance ratio, $\Delta V$ the velocity interval and $dv/dr$ is the velocity gradient. The CO culumn density is degenerated with the beam filling factor, the velocity gradient and the abundance ratio. Since we didn't correct our observed line intensities with beam filling factors, the derived CO column densities in each velocity bin should be considered as lower limits. As shown in Figure 3 of \citet{2009ApJ...696...66Q}, the source size are $\sim 20\arcsec$ and $\sim 10\arcsec$ at $\sim \pm$ 7 km s$^{-1}$ and $\sim \pm$ 22 km s$^{-1}$ with respect to the cloud velocity, corresponding to beam filling factors of $\sim$ 0.5 and $\sim$ 0.2, respectively.  From the lowest velocities to the highest velocities of the outflowing gas, the beam-averaged CO column density has decreased 25 times, whereas the beam filling factor has only decreased 2.5 times. So one or serveral of these parameters ($n_{\mathrm{H}_2}$, $dv/dr$ and $X_{\mathrm{CO}}$) related to the CO culumn density must have decreased. Due to the lack of further informations, it is unclear whether the CO abundance ratio or the velocity gradient has attributed to the decrease of CO column density at higher velocities. If we assume the CO abundance ratio and the velocity gradient to be constant, the decrease of CO column density could be interpreted by a decrease of gas density with velocity. This decreasing density-velocity trend is consistent with the wind-driven models which predict that the wind density decreases with velocity and distance from the driving source \citep{2007prpl.conf..245A}.
%A drop in $N_{\mathrm{CO}}$ at more extreme velocities indicates the decrease of one or serveral of these parameters. 
%To explore how the effect of beam dilution influence our results, we vary the beam filling factors from 0.2 to 1 and then perform the LVG analysis again. We find that modelling with different beam filling factors mainly affect the $N$-$V$ diagram, with minor change in the $T$-$V$ diagram and density limits. This could be resulted from the degeneracies of the beam filling factor with CO column density in the optically thin case. 