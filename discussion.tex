\section{DISCUSSION}\label{discussion}
Since the molecular outflow from YSOs was discovered, several outflow models on explaining how outflows are driven have been proposed \citep{2007prpl.conf..245A}. Among these models, the wide-angle wind-driven model \citep{1991ApJ...370L..31S,1996ApJ...472..211L, 2001ApJ...557..429L} and the jet-driven bow shock model \citep{ 1993A&A...278..267R, 1993ApJ...414..230M, 2001ApJ...557..429L} have been proven to be the most promising models.
In the wind-driven model, a molecular outflow is the ambient material swept-up by a wide-angle radial wind. In the jet-driven bow shock model, a molecular outflow is accelarated by a jet bow shock when a jet propagates into the ambient gas. Though these models predict many different outflow properties \citep{2007prpl.conf..245A}, most of the previous observational work has focused on comparing the morphology and the kinematics of outflows with the model predictions. The shapes of the molecular outflows range from parabolic molecular shells in RNO 91 and VLA 05487 \citep{2000ApJ...542..925L}, which is consistent with the wind-driven model, to the bow-shaped shells in HH211 \citep{1999A&A...343..571G} and HH 212 \citep{2000ApJ...542..925L}, which is consistent with the jet-driven model. There are also outflows where the two diffferent features exist simultaneously (HH 111, \citet{2000ApJ...542..925L}; HH 46/47, \citet{2013ApJ...774...39A}, \citet{2016ApJ...832..158Z}; IRAS 04166+2706, \citet{2009A&A...495..169S}).
Between the wind-driven model and the jet-driven model, their predictions on the P-V diagrams cut along the outflow axis can be clearly distinguished: a convex spur structure with the highest velocity at the bow tip as predicted by the jet-driven model and a parabolic structure originating from the central source as predicted by the wind-driven model \citep{2001ApJ...557..429L}. The P-V relations of some CO outflows are best explained by the jet-driven model, such as the outflows of HH 212 and HH 240/241 \citep{2000ApJ...542..925L}. And the P-V relations of some other CO outflows are best explained by the wind-driven model, such as the outflows drived by some low-mass YSOs (e.g. IRAS 16293-2422, \citet{2008ApJ...675..454Y}; VLA 05847+0255, \citet{2000ApJ...542..925L}) and high-mass YSOs (e.g. G240.31+0.07, \citet{2009ApJ...696...66Q}). While the P-V structure of some CO outflows need to be explained by a combination of a jet-driven model and a wind-driven model, such as the HH 111 outflow \citep{2000ApJ...542..925L} and the HH 46/47 outflow \citep{2013ApJ...774...39A}. Each model also show a difference between the power-law index of the M-V relation. However, different simulations of the same model have yielded different slopes for the computed CO intensity-velocity relations, making it difficult to compare the observed power-law index with the model predictions. For a outflow drived by a low-mass YSO, the wide-wind models generally predict a smaller intensity-velocity power-law index over a narrower range when compared to the jet models \citep{2007prpl.conf..245A}.

\subsection{Temperature}

The $T-V$ relation of outflows have been studied by several authors.  A rising CO 3-2/6-5 ratio is observed towards the outflow associated with low-mass YSO HH 46 IRS 1\citep{2009A&A...501..633V}. With the density assumed to be constant, the rising ratios at more extreme velocities correspond to lower kinetic temperatures. In the case of the outflow associated with low-mass protostars NGC 1333 IRAS 4A and  IRAS 4B, the CO 3-2/6-5 ratios are remarkably constant with velocity \citep{2012A&A...542A..86Y}. With the assumption of constant density, the constant ratios indicate constant temperatures. \citet{2012ApJ...744L..26S} have imaged the extremely high velocity (EHV) outflow in CO (2-1) and CO (3-2) associated with the high-mass YSO G5.89-0.39. With the assumption of a canonical CO fractional abundance of 10$^{-4}$, an increasing trend of temperature with outflow velocity is found by performing a LVG analysis. Using the CO (6-5), (7-6) and (16-15) lines, \citet{2015A&A...584A..70L} performed a RD analysis towards the G5.89-0.39 outflow and found a decreasing trend of temperature with increasing velocities. This disagreement seen in results of \citet{2012ApJ...744L..26S} and \citet{2015A&A...584A..70L} could be due to different angular resolutions (3${\arcsec}$.4 compared to 14${\arcsec}$.5 ). The different $T-V$ relations found in different outflows and in different angular scales reveal the complexity of molecular outflows. However, with only two or three lines observed, their derived $T-V$ relations must be deduced from priori assumptions of other physical parameters, e.g., constant density or canonical CO fractional abundance, or from theoretical considerations, e.g., LTE. To get more proper estimations of the pysical parameters of the molecular outflow, multi-line studies of CO are needed.

The LVG analysis and the RD analysis reveal that the G240 outflow is isothermal and has a temperature of $\sim$ 50 K . This value is consistent with temperatures in excess of 50 K probed by \citet{2016A&A...587A..17V} for outflows associated with intermediate-mass protostars, and slightly lower than temperatures of outflows associated with low-mass protostars \citep{2009A&A...501..633V, 2012A&A...542A..86Y}. The isothermal state rules out the jet-driven bow shock models, which predicts temperature rising with outflow velocity and distance, and provides evidence for the wind-driven model \citep{2007prpl.conf..245A}. As molecular cooling dominates the cooling of the shocked material in the outflow at temperatures below 10$^4$ K \citep{1997IAUS..182..181H} and the cooling rate increases as $n^2$, molecular cooling is very efficient for the density of a wind-driven outflow. Thus, an isothermal state could be reached in a wind-driven outflow. We noticed that there are faint bow-shaped H$_{2}$ features near the YSO \objectname{IRAS 07427-2400}, suggesting that the jet-driven outflows may also exist. However, as the average T-V relation of the outflowing gas agrees with the wind-driven model and the kinematics and morphology of the molecular outflow can also be qualitatively interpreted by the wide-angle wind-driven model \citep{2009ApJ...696...66Q}, we conclude that even if the jet-driven outflows and the wind-driven outflows were co-existing in the G240 outflow, the wide-angle wind entrainment plays a more important role in driving the G240 outflow. It should be noted that, most existing outflow models have parameters typical of low-mass outflows. It is necessary to compare the observational results of high-mass outflows with models of similar physical conditions. Statics of outflows associated with high-mass star-forming regions are also essential for us to better understand the driven mechanism of massive outflows and the forming process of high-mass stars.

\subsection{Density}

The LVG analysis reveals that the lower limits of gas densities are $\sim 10^5$ cm$^{-3}$ at the velocities where $\chi^2_{\mathrm{red}} \sim 1$. We have found a decreasing trend of the beam averaged CO column density with gas velocity. As shown in the $N$-$V$ diagram of Figure \ref{fig:fig4}, for each velocity bin (2 km s$^{-1}$), the beam averaged CO column density drops from $\sim 2 \times  10^{16} $ cm$^{-2}$ to $\sim 8 \times 10^{14}$ cm$^{-2}$ within 15 km s$^{-1}$. In the optically thin case, the beam averaged CO column density could be related to gas density $n_{\mathrm{H}_2}$: 
\begin{equation}
N_{\mathrm{CO}} = n_{\mathrm{H}_2} \times \Delta V \times \frac{1}{dv/dr} \times X_{\mathrm{CO}} \times f_{\mathrm{b}}, 
\end{equation}
where $f_{\mathrm{b}}$ is the beam filling factor, $X_{\mathrm{CO}}$ the CO/H$_2$ abundance ratio, $\Delta V$ the velocity interval and $dv/dr$ is the velocity gradient. A drop in $N_{\mathrm{CO}}$ at more extreme velocities indicates the decrease of one or serveral of these parameters. 
To explore how the effect of beam dilution influence our results, we vary the beam filling factors from 0.2 to 1 and then perform the LVG analysis again. We find that modelling with different beam filling factors mainly affect the $N$-$V$ diagram, with minor change in the $T$-$V$ diagram and density limits. This could be resulted from the degeneracies of the beam filling factor with CO column density in the optically thin case. 

As shown in Figure 3 of \citet{2009ApJ...696...66Q}, the source size are $\sim 20\arcsec$ and $\sim 10\arcsec$ at the velocities of $\sim \pm$ 6 km s$^{-1}$ and $\sim \pm$ 20 km s$^{-1}$ with respect to the cloud velocity, corresponding to beam filling factors of $\sim$ 0.5 and $\sim$ 0.2, respectively. Considering the 2.5 times drop in the beam filling factor, the 25 times drop in the beam averaged CO column density indicates $\sim$ 10 times decrease in the CO column density. Due to the lack of other informations, we cannot assess whether the CO abundance ratio or the velocity gradient has attributed to the drop of CO column density at high velocities. As shown in \citet{2007prpl.conf..245A}, the wind-driven models predict that the wind density decreases with velocity and distance from the driving source. Thus, if we assume the CO abundance ratio and the velocity gradient to be constant, the decrease of CO column density could be interpreted by a decrease of gas density with velocity.