\section{OBSERVATIONS}
The Submillimeter Array\footnote{    The Submillimeter Array is a joint project between the Smithsonian Astrophysical Observatory and the Academia Sinica Institute of Astronomy and Astrophysics and is funded by the Smithsonian Institution and the Academia Sinica.} (SMA) observations were carried out on 2008 February 25 with eight antennas in the compact configuration and on 2008 February 16 with seven antennas in the extended configuration. To ensure the coverage of the entire outflow we observed two fields centered at (R.A., del.)$_{J2000}$ = (07$^\textup{h}$44$^\textup{m}$52$^\textup{s}$.49, −24$^\textup{d}$07$^\textup{m}$52$^\textup{s}$.1) and (R.A., del.)$_{J2000}$ = (07$^\textup{h}$44$^\textup{m}$51$^\textup{s}$.07, −24$^\textup{d}$07$^\textup{m}$34$^\textup{s}$.9), respectively. We used Titan as the primary flux calibrator and \object{3c273} as the bandpass calibrator. The time dependent gain was monitored
by observing quasars \object{J0730-116} and \object{J0826-225} every 23 minutes. Visibilities were calibrated using the IDLMIR package and then output to MIRIAD for imaging. With natural weighting the synthesized beams in the compact and extended configurations are about 4${\arcsec}$ $\times$ 3${\arcsec}$ and 1${\arcsec}$.2 $\times$ 1${\arcsec}$, respectively. The shortest baseline in our SMA observations is 16.5 m, corresponding to a spatial scale of 20 ${\arcsec}$ for observations at 225 GHz. Thus, spatial structures more extended than 20 ${\arcsec}$ were not sampled in the SMA observations. This spatial filtering can significantly affect the CO (2-1) maps at velocities close to the cloud velocity. To recover the missing short spacing information we observed the CO (2-1) emission with the Caltech
Submillimeter Observatory\footnote{    The Caltech Submillimeter Observatory was supported by the NSF grant
AST-0229008 and was closed on September 18, 2015.} (CSO) 10.4 m telescope on 2008 February 12. During the observation the weather condition was excellent for 1 mm waveband with $\tau_{225GHz}$ $\approx$ 0.08. The observations were carried out in the on-the-fly mode centered on (R.A., del.)$_{J2000}$ = (07$^\textup{h}$44$^\textup{m}$52$^\textup{s}$.1, −24$^\textup{d}$07$^\textup{m}$49$^\textup{s}$). We obtained a 15 × 15 grid in CO (2-1), with an integration of about 5 s on each cell. The 10${\arcsec}$ cell spacing is about one-third of the CSO beam, which is $\approx$ 32${\arcsec}$.5 in CO (2-1). By observing Mars and Saturn we derived a beam efficiency of 0.58 ± 0.10. The spectrometer used has 1024 channels throughout the 50 MHz bandwidth, resulting in a spectral resolution of 0.0488 MHz (about 0.065 km s$^{-1}$) per channel. The final maps were smoothed to 2 km s$^{-1}$ per channel. The data were reduced using the standard CLASS package. We combined the SMA compact and CSO CO (2-1) data in MIRIAD following a procedure outlined in \citet{1995ApJ...451L..71Z}. 

The 12-m submillimeter Atacama Pathfinder Experiment Telescope\footnote{    APEX is a collaboration between the Max-Planck-Institut f"ur Radioastronomie, the European Southern Observatory, and the Onsala Space Observatory.} (APEX) observations were conducted on . CO (6-5) and CO (7-6) were observed simultaneously. 