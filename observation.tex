\section{Observations}

The observations were performed with the 12 m APEX telescope. The APEX CO J = 6-5 and CO J = 7-6 observations were
performed in 2010 July with the Carbon Heterodyne Array of the MPIfR (CHAMP+) \citep{2006SPIE.6275E..0NK}. The APEX CO (3-2) observations were made in 2011 October using the First Light APEX Submillimeter Heterodyne (FLASH) receiver \citep{2006A&A...454L..21H}.

Pointings were checked by comparing these lines with the CO J = 2-1 data adopted from \citet{2009ApJ...696...66Q}, and were found to be within $\sim 7\arcsec$ for CO J = 3-2 and within $\sim 4\arcsec$ for CO J = 6-5 and J = 7-6. The system temperatures were found to be ? at ? GHz. The ? spectrometer was used as the backend with a resolution of ? kHz (? km s$^{-1}$). The data were smoothed to 2 km s$^{-1}$, then the rms noises in the spectra in the central part of the map are of the order 0.03-0.05 K at 345 GHz, 0.15-0.25 K at 691 GHz and 0.40 - 0.60 K at 809 GHz at the central region of the maps. Noises are higher at the edge of the maps. Beam efficiencies, determined by observations of planets, were 0.65, 0.41 and 0.40 at 345 GHz, 691 GHz and 809 GHz, respectively. The APEX beam sizes are $\sim 19\arcsec$ at 345 GHz, $\sim 9\arcsec$ at 691 GHz and $\sim 8\arcsec$ at 809 GHz. Calibration uncertainties were assumed to be 15 \%, 20\% and 30\% at 345 GHz, 691 GHz and 809 GHz, relatively. These data were complemented with the perviously observed combined data from the Submillimeter Array\footnote{    The Submillimeter Array is a joint project between the Smithsonian Astrophysical Observatory and the Academia Sinica Institute of Astronomy and the Astrophysics and is funded by the Smithsonian Institution and the Academia Sinica.} (SMA) and the Caltech Submillimeter Observatory\footnote{    The Caltech Submillimeter Observatory was supported by the NSF grant AST-0229008 and was decommissioned in 2015.} (CSO) 10.4 m telescope in CO J = 2-1 \citep{2009ApJ...696...66Q}. The CO J = 2-1 data has a calibration uncertainty of 10 \%, with the rms estimated to be 0.06-0.10 K in 2 km s$^{-1}$ channels. Table \ref{tab:lines} shows a summary of the line informations.

\begin{deluxetable}{ccCccc}[t!]
\tablecaption{Line informations \label{tab:lines}}
\tablecolumns{6}
\tablewidth{0pt}
\tablehead{
\colhead{CO line} &
\colhead{Frequency} &
\colhead{Beam size} & 
\colhead{$\sigma_{cal}$ \tablenotemark{a}} & 
\colhead{$\sigma_{rms}$\tablenotemark{b}} & 
\colhead{$\eta_s$\tablenotemark{c}} \\
\colhead{} & \colhead{GHz} &
\colhead{$\arcsec$} & \colhead{\%} & \colhead{K} & \colhead{}
}
\startdata
2-1 & 230.5380 & 3.93 $\times$ 3.10\tablenotemark{d} & 10 & 0.08& $-$\\
3-2 & 345.7960 & 19.16 & 15 & 0.04 & 0.65 \\
6-5 & 691.4731 & 9.49 & 20 & 0.20 & 0.41 \\
7-6 & 806.6518 & 8.12 & 30 & 0.50 & 0.40 \\
\enddata
\tablenotetext{a}{Calibration error.}
\tablenotetext{b}{Average rms in 2 km s$^{-1}$ channels.}
\tablenotetext{c}{Beam efficiency.}
\tablenotetext{d}{Major axis $\times$ minor axis.}
\end{deluxetable}
